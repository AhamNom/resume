%%%%%%%%%%%%%%%%%%%%%%%%%%%%%%%%%%%%%%%%%
% Medium Length Professional CV
% LaTeX Template
% Version 2.0 (8/5/13)
%
% This template has been downloaded from:
% http://www.LaTeXTemplates.com
%
% Original author:
% Trey Hunner (http://www.treyhunner.com/)
%
% Important note:
% This template requires the resume.cls file to be in the same directory as the
% .tex file. The resume.cls file provides the resume style used for structuring the
% document.
%
%%%%%%%%%%%%%%%%%%%%%%%%%%%%%%%%%%%%%%%%%

%----------------------------------------------------------------------------------------
%	PACKAGES AND OTHER DOCUMENT CONFIGURATIONS
%----------------------------------------------------------------------------------------

\documentclass{resume} % Use the custom resume.cls style
\usepackage[UTF8]{ctex}
\usepackage[left=0.75in,top=0.6in,right=0.75in,bottom=0.6in]{geometry} % Document margins
\usepackage{xeCJK}  
\usepackage{color}
\usepackage[citecolor=green
            ]{hyperref}
\usepackage{hyperref}

\name{林凯锋} % Your name
\address{+(86) 18759188810 \\ \href{mailto:D_5254@hotmail.com}{D\_5254@hotmail.com}} % Your phone number and email
% \address{22 岁, 男性}  % Your address
\begin{document}

%----------------------------------------------------------------------------------------
%	EDUCATION SECTION
%----------------------------------------------------------------------------------------

\begin{rSection}{教育经历}
{\textbf{福建理工大学}} \hfill {\em 2020.9 - 2024.6} \\ 
计算机科学与数学学院 - 计算机科学与技术

\end{rSection}

\begin{rSection}{校园经历}

\begin{rSubsection}{ACM 算法竞赛集训队}{\em 2020.12 - 2022.11}
    {}
    {}
    \item[]
    \begin{itemize}
        \setlength\itemsep{-0.5em}
        \item[-] 学习算法,大量刷题, 训练思维、训练复杂代码的快速编写,参加 ACM 竞赛。多次获得\textbf{ ACM 区域赛铜奖}。
        \item[-] 擅长\textbf{计算几何}、\textbf{字符串}、\textbf{组合计数}算法。
    \end{itemize}
\end{rSubsection}

\end{rSection}

%----------------------------------------------------------------------------------------
%	HONORS / AWARDS SECTION
%----------------------------------------------------------------------------------------

\begin{rSection}{奖项}
\begin{tabular}{ @{} l @{\hspace{6ex}} l }
CCSP 中国大学生计算机系统与程序设计竞赛华东赛区 \textbf{铜奖} & {\em 2021.12} \\
ACM-ICPC 国际大学生程序设计竞赛亚洲区域赛 \textbf{铜奖} & {\em 2022.11} \\
CCPC 中国大学生程序设计竞赛 \textbf{铜奖} & {\em 2022.11} \\
\end{tabular}
\end{rSection}

%----------------------------------------------------------------------------------------
%	WORKING EXPERIENCE SECTION
%----------------------------------------------------------------------------------------

\begin{rSection}{实习经历}

\begin{rSubsection}{朴朴科技}{\em 2023.7 - 2023.9}
    {\textit{实习生,大数据开发工程师}}
    {\textit{}}
    \item[]
    \begin{itemize}
        \setlength\itemsep{-0.5em}
        \item[-] 参与数据 ETL, 离线及实时数据仓库的设计及开发工作。
        \item[-] 完善数据仓库的监控和告警。
        \item[-] 优化数据仓库的性能, 降低查询时间和膨胀倍数。
    \end{itemize}
\end{rSubsection}

\end{rSection}

%----------------------------------------------------------------------------------------
%	PROJECTS / RESEARCH EXPERIENCE SECTION
%----------------------------------------------------------------------------------------

\begin{rSection}{项目}

\begin{rSubsection}
    {\textbf{面向算法竞赛的评测系统和编译器优化}}{\em 2023.1 - 至今}
    {\textit{容器化部署开源的比赛系统 DOMjudge, 并修改 Clang 编译器行为实现优化。}\\}
    {\textit{可在集训队内部日常训练中使用, 为队员训练节约大量时间。作为我的毕业设计。}}
    \item[]
    \begin{itemize}
        \setlength\itemsep{-0.5em}
        \item[-] 评测系统优化: 将常用算法封装至可直接使用, 减小入门选手的原理学习门槛, 及进阶选手的重复工作和低级错误。设置自己编写的\textbf{微型编译器前端}可供选择, 能避免一些麻烦的细节如 lambda 的递归 (改能直接在函数中定义函数), 扩展了一些运算符。
        \item[-] 编译器优化: 通过 \textbf{Clang plugin 和 LLVM Pass} 的编写, 实现了多种算法的自动优化。包括: 对矩阵线段树优化, 识别和去除被嵌套向量和矩阵中不必要的元素, 以此优化矩阵乘法; 对\href{https://oi-wiki.org/math/poly/linear-recurrence/}{常系数齐次线性递推}优化, 使用矩阵快速幂+快速傅里叶变换可以将复杂度从 $O(nk)$ 优化至 $O(klogklogn)$; 对表达式的取值范围进行静态分析, 实现下标访问范围的提取, 以此由编译器自动插入一些打表 (质数、阶乘、阶乘逆元等); 通过读取特定的属性标记预知元素的取值范围, 以此对选手代码中的数据结构进行针对性优化; 静态分析程序中的 STL 容器使用情况, 插入实现了\textbf{内存池}的分配器减少系统调用开销。
        \item[-] 计划实现的编译器功能: 对动态规划中是否有 "从未初始化的状态转移" 进行静态检查; 对部分数学题由代码逆向 \LaTeX; 通过新增关键字实现改变局部变量所处的栈帧 (不在本函数的栈帧), 结合 C++ 17 的 NRVO 特性达到减少拷贝的效果。
    \end{itemize}
\end{rSubsection}

\begin{rSubsection}
	{7z to zip}{\em 2023.3}
	{\textit{一个 Linux 命令行工具。将目录和子目录下的 7z 文件以 zip 压缩到目标路径, 且保持原始路径结构。}}
	{\textit{}}
	\item[]
	\begin{itemize}
		\setlength\itemsep{-0.5em}
		\item[-] 编写\textbf{线程安全且无锁的消息队列}, 实现了日志类。
		\item[-] 使用 C++ 20 的并发支持库编写多线程解压缩和压缩。
		\item[-] 实现了 \textbf{CPU 亲和}及 \textbf{NUMA 亲和},并对内存占用进行了优化。
		\item[-] 程序实现了高度的可配置性, 能从命令行参数和 yaml 类型的配置文件中读取多种信息。
	\end{itemize}
\end{rSubsection}

\begin{rSubsection}
    {AhamBT (回测框架)} {\em 2023.10 - 2023.12}
    {\textit{量化交易所需的事件驱动回测系统框架, 支持基于 tick 数据的低延时策略。}}
    {\textit{}}
    \item[]
    \begin{itemize}
        \setlength\itemsep{-0.5em}
        \item[-] 项目使用 \textbf{C++ 23} 编写, 使用 xmake 构建, 使用 modules 特性使得项目结构清晰。
        \item[-] 基于 nanolog 实现了\textbf{超低延时}的日志记录。了解到 nanolog 的真无锁 (无 CAS) 和高缓存命中等机制的实现原理。
        \item[-] 通过树形结构支持来自多策略的复合策略, 有很强的可复用性, 能够地灵活对比不同策略组合。
    \end{itemize}
\end{rSubsection}
\end{rSection}

%----------------------------------------------------------------------------------------
%	SKILLS SECTION
%----------------------------------------------------------------------------------------

\begin{rSection}{技能}
\begin{rSubsection}
{}{}{}{}
\item[-] 语言: 精通 C++ 语法并且熟悉 \textbf{C++ 11 至 C++ 23 各版本新特性}。熟悉模板元编程。熟悉 modules、coroutines、ranges、concepts 等重要的现代 C++ 特性。了解 Python 语言。
\item[-] 原理: 了解 \textbf{Clang 编译器源码}。了解 \textbf{Docker 容器原理与实现}。了解\textbf{内存池和 NUMA 相关优化}。熟悉 TCP、HTTP、QUIC、WebSocket 等网络协议的基本原理。
\item[-] 工具: 熟悉 git 工具及其原理。熟悉 perf 性能分析、lldb 调试。熟悉 makefile、cmake 和 xmake 等构建工具。有 Docker 容器使用经验。熟悉 shell、awk 等常用脚本语言。
\item[-] 英语等级:\textbf{六级}通过。
\end{rSubsection}
\end{rSection}

%----------------------------------------------------------------------------------------
%	OTHERS SECTION
%----------------------------------------------------------------------------------------

\end{document}